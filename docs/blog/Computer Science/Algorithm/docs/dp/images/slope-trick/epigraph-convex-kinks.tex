% Modified from https://tex.stackexchange.com/a/261567.
\documentclass{standalone}
\usepackage[active,tightpage]{preview}
\usepackage{amsmath}
\usepackage{pgfplots}
\usepgfplotslibrary{fillbetween}
\usetikzlibrary{arrows.meta,intersections}

\usepackage{anyfontsize}
\renewcommand{\normalsize}{\fontsize{14pt}{16pt}\selectfont}

\DeclareMathOperator{\dom}{dom}

\pgfmathdeclarefunction{func}{1}{%
  \pgfmathparse{max(19-4*(#1+8),19-3*(#1+8),7-1*(#1+4),2,2+1*(#1-3),3+2*(#1-4),7+3*(#1-6),19+4*(#1-10))}%
}

\pgfplotsset{compat=1.16}
\makeatletter
\pgfplotsset{
    mark max/.style={
        point meta rel=per plot,
        visualization depends on={x \as \xvalue},
        scatter/@pre marker code/.code={%
        \ifx\pgfplotspointmeta\pgfplots@metamax
            \def\markopts{mark=none}%
            \coordinate (maximum);
        \fi
            \def\markopts{mark=none}
            \expandafter\scope\expandafter[\markopts]
        },%
        scatter/@post marker code/.code={%
            \endscope
        },
        scatter
    }
}

% Syntax
% \DrawEpigraph[<additional options>]{<min domain x>}{<max domain x>}{<shift on the left>}{<shift on the right>}
\newcommand\DrawEpigraph[6][draw=white,top color=red!05,bottom color=red!45]{
  \coordinate (plot-left) at ([yshift=#4]axis cs:#2,\pgfplots@metamax);
  \coordinate (plot-right) at ([yshift=#5]axis cs:#3,\pgfplots@metamax);
  \path[name path=diagonal,draw=none] (plot-left) -- (plot-right);
  \addplot[#1] fill between[of=#6 and diagonal];
}
\makeatother

\PreviewEnvironment{tikzpicture}

\begin{document}        

\begin{tikzpicture}
\begin{axis}[
  axis lines=middle,
  xlabel={$x$},
  ylabel={$y$},
  xtick={\empty},
  ytick={\empty},
  domain=-8.2:10.2,
  ymin=0,
  ymax=20,
  xmin=-8.5,
  xmax=11,
  clip=false,
  axis on top,
  scale=2
] 
% The curve
\addplot [mark max,black,thick,name path=B,samples=1000] plot {func(x)};
% The Epigraph
\DrawEpigraph{-8.2}{10.2}{0}{0}{B}
% Lines and labels
\node[below left] at (axis cs:0,0) {$O$};
\draw[dashed] 
    (axis cs:-8,0) node[below] {$\xi_{-4}$} -- (axis cs:-8,{func(-8)})
    node[draw,fill,circle,inner sep=-1.5pt] {};
\draw[dashed] 
    (axis cs:-4,0) node[below] {$\xi_{-3}=\xi_{-2}$} -- (axis cs:-4,{func(-4)})
    node[draw,fill,circle,inner sep=-1.5pt] {};
\draw[dashed] 
    (axis cs:1,0) node[below] {$\xi_{-1}$} -- (axis cs:1,{func(1)})
    node[draw,fill,circle,inner sep=-1.5pt] {};
\draw[dashed] 
    (axis cs:3,0) node[below] {$\xi_1$} -- (axis cs:3,{func(3)})
    node[draw,fill,circle,inner sep=-1.5pt] {};
\draw[dashed] 
    (axis cs:4,0) node[below] {$\xi_2$} -- (axis cs:4,{func(4)})
    node[draw,fill,circle,inner sep=-1.5pt] {};
\draw[dashed] 
    (axis cs:6,0) node[below] {$\xi_3$} -- (axis cs:6,{func(6)})
    node[draw,fill,circle,inner sep=-1.5pt] {};
\draw[dashed] 
    (axis cs:10,0) node[below] {$\xi_4$} -- (axis cs:10,{func(10)})
    node[draw,fill,circle,inner sep=-1.5pt] {};
\draw[dashed] 
    (axis cs:0,{func(1)}) node[left] {$\min f(x)$} -- (axis cs:1,{func(1)})
    node[draw,fill,circle,inner sep=-1.5pt] {};
\node[right] at (axis cs:-6,{func(-6)}) {$k=-3$};
\node[right] at (axis cs:-2.8,{func(-2.8)}) {$k=-1$};
\node[above] at (axis cs:2,{func(2)}) {$k=0$};
\node[right] at (axis cs:3,{func(3)}) {$k=1$};
\node[right] at (axis cs:4.5,{func(4.5)}) {$k=2$};
\node[right] at (axis cs:7.5,{func(7.5)}) {$k=3$};
\end{axis}
\end{tikzpicture}

\end{document}
